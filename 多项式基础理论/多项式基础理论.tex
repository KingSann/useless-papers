%!TEX program = xelatex

\documentclass[UTF8]{article}
\author {sjzez czy}
\title {多项式基础理论}
\date{2019.2.18}
\usepackage[UTF8]{ctex}

\usepackage{listings}
\usepackage{fontspec}
\usepackage{amsmath}
\usepackage{amssymb}
\usepackage{geometry}
\usepackage{setspace}
\usepackage{abstract}
\usepackage{graphicx}
\usepackage{verbatim}
\usepackage[colorlinks,linkcolor=black,citecolor=black]{hyperref}
\renewcommand{\baselinestretch}{1.5}
\geometry{a4paper,left=2.7cm,right=2.7cm,top=2.7cm,bottom=2.7cm}
\setmonofont{Consolas}

\begin{document}

\maketitle

\tableofcontents

\newpage

\section{多项式求逆}

\begin{quotation}
    给定 $f(x)$,求 $g(x)$ 满足:

    $$
    f(x) \times g(x) \equiv 1 \pmod {x^{n+1}}
    $$
\end{quotation}

设:

$$
\begin{cases}
A(x)B(x) \equiv 1 \pmod {x^n} \\
A(x)C(x) \equiv 1 \pmod {x^\frac{n}{2}} \\
\end{cases}
$$

则有:

$$
\begin{aligned}
& A(x)B(x) \equiv 1 \pmod {x^{\frac{n}{2}}} \\
\Rightarrow &A(x)(B(x)-C(x)) \equiv 0 \pmod {x^{\frac{n}{2}}} \\
\Rightarrow &B(x)-C(x) \equiv 0 \pmod {x^{\frac{n}{2}}} \\
\Rightarrow &(B(x)-C(x))^2 \equiv 0 \pmod {x^n} \\
\Rightarrow &B^2(x)+C(x)^2-2B(x)C(x) \equiv 0 \pmod {x^n} \\
\Rightarrow &A(x)B^2(x)+A(x)C(x)^2-2A(x)B(x)C(x) \equiv 0 \pmod {x^n} \\
\Rightarrow &B(x)+A(x)C(x)^2-2C(x) \equiv 0 \pmod {x^n} \\
\Rightarrow &B(x) \equiv 2C(x)-A(x)C(x)^2 \pmod {x^n}
\end{aligned}
$$

可见一个多项式是否有逆元只与其常数项是否有逆元有关

\section{一阶分治FFT转多项式求逆}

\begin{quotation}

    已知 $\{g_{i} | i \in [1,n-1] \cap Z\}$,且 $f_0=1$,同时有 $f_i=\sum_{j=1}^{i}f_{i-j}g_j$

    求 $\{ f_i | i \in [0,n - 1] \cap Z \}$
\end{quotation}

设 $F(x)=\sum_{i=0}^{\infty}f_ix^i,G(x)=\sum_{i=0}^{\infty}g_ix^i$,且 $g_0=0$

那么有:

$$
\begin{aligned}
& F(x)G(x)=\sum_{i=0}^{\infty}x^i\sum_{j+k=i}f_jg_k=F(x)-f_0x^0 \\
\Rightarrow & F(x)G(x) \equiv F(x)-f_0 \pmod {x^n} \\
\Rightarrow & F(x) \equiv \frac{f_0}{1-G(x)} \pmod {x^n} \\
\Rightarrow & F(x) \equiv (1-G(x))^{-1} \pmod {x^n} \\
\end{aligned}
$$

\section{多项式除法}

\begin{quotation}
    给定 $n$ 次多项式 $f(x)$ 和 $m$ 次多项式 $g(x)$,求 $n-m$ 次多项式 $h(x)$ 和一个 $m-1$ 次多项式 $r(x)$,满足:

    $$
    f(x) \equiv g(x) \times h(x)+r(x) \pmod {x^{n+1}}
    $$
\end{quotation}

设一个 $n$ 次多项式的系数翻转为:

$$
\hat f(x)=x^nf(\frac{1}{x})
$$

同时也可以得到它的逆翻转:

$$
f(\frac{1}{x})=x^{-n}\hat f(x)
$$

考虑如下变形:

$$
\begin{aligned}
&f(x) \equiv g(x) \times h(x)+r(x) \pmod {x^{n+1}} \\
\Rightarrow 
& f(\frac{1}{x}) \equiv g(\frac{1}{x}) \times h(\frac{1}{x})+r(\frac{1}{x}) \pmod {x^{n+1}} \\
\Rightarrow 
& x^{-n}\hat f(x) \equiv x^{-m} \hat g(x) \times x^{m-n}\hat h(x)+x^{1-m}\hat r(x) \pmod {x^{n+1}} \\
\Rightarrow 
& \hat f(x) \equiv \hat g(x) \times \hat h(x)+x^{n-m+1}\hat r(x) \pmod {x^{n+1}} \\
\Rightarrow 
& \hat f(x) \equiv \hat g(x) \times \hat h(x) \pmod {x^{n-m+1}} \\
\Rightarrow 
& \hat h(x) \equiv \frac{\hat f(x)}{\hat g(x)} \pmod {x^{n-m+1}} \\
\end{aligned}
$$

于是可以求出 $h(x)$,之后有:

$$
r(x) \equiv f(x)-g(x) \times h(x) \pmod {x^{m}}
$$

\section{多项式牛顿迭代}

    给定 $f(x)$,求一个 $g(x)$,满足:
    
    $$
    f(g(x)) \equiv 0 \pmod {x^{n+1}}
    $$

设 $h(g)=f(g)$,假设已经求得了:

$$
h(g_0) \equiv 0 \pmod {x^{\frac{n}{2}}}
$$

现在要求 $g$ 使得:

$$
h(g) \equiv 0 \pmod {x^{n}}
$$

考虑 $h(g)$ 在 $g_0$ 处的麦克劳林展开:

$$
\begin{aligned}
&h(g) \equiv \sum_{n = 0}^{\infty} \frac{h^{(n)}(g_0)(g-g_0)^n}{n!} \pmod {x^{n}} \\
\Rightarrow
&0 \equiv h(g) \equiv h(g_0)+h'(g_0)(g-g_0) \pmod {x^{n}} \\
\Rightarrow
&g \equiv g_0-\frac{h(g_0)}{h'(g_0)} \pmod {x^{n}}
\end{aligned}
$$

\section{多项式ln}

\begin{quotation}
    给定 $f(x)$,求 $g(x) \equiv \ln(f(x)) \pmod {x^{n+1}}$
\end{quotation}

左右求导后可以得出:

$$
\begin{aligned}
& g'(x) \equiv \frac{f'(x)}{f(x)} \pmod {x^{n+1}} \\
\Rightarrow 
& g(x) \equiv \int \frac{f'(x)}{f(x)}dx \pmod {x^{n+1}}
\end{aligned}
$$

\section{多项式exp}

\begin{quotation}
    给定 $f(x)$,求 $g(x) \equiv e^{f(x)} \pmod {x^{n+1}}$
\end{quotation}

先左右取对数:

$$
\begin{aligned}
& \ln(g(x)) \equiv f(x) \pmod {x^{n+1}} \\
\Rightarrow
&h(g) \equiv \ln(g)-f \equiv 0\pmod  {x^{n+1}}
\end{aligned}
$$

之后对 $h(g)$ 进行牛顿迭代:

$$
\begin{aligned}
& g \equiv g_0-\frac{h(g_0)}{h'(g_0)} \pmod {x^{n}} \\
\Rightarrow 
& g \equiv g_0-h(g_0)g_0 \pmod {x^{n}} \\
\Rightarrow 
& g \equiv (1-\ln(g_0)+f)g_0\pmod {x^{n}} \\
\end{aligned}
$$

\section{多项式多点求值}

\begin{quotation}
    给定 $f(x)=\sum_{i=0}^{n}a_ix^i$,同时有 $n+1$ 个点值 $X=\{x_i | 0 \le i \le n\}$
    
    求 $Y=\{f(x) | x \in X\}$
\end{quotation}

构造函数 $X_l(x)=\prod_{i=0}^{\lfloor \frac{n}{2} \rfloor}(x-x_i),X_r(x)=\sum_{i=\lfloor \frac{n}{2} \rfloor+1}^{n}(x-x_i)$

设 $f(x)=A_l(x)X_l(x)+B_l(x)$,则 $\forall x \in X_l,f(x)=B_l(x)$

设 $f(x)=A_r(x)X_r(x)+B_r(x)$,则 $\forall x \in X_r,f(x)=B_r(x)$

换句话说,可以求得 $B_l(x)=f(x) \bmod A_l(x)$,然后递归处理 $[l,\lfloor \frac{n}{2} \rfloor]$,右区间同理

在到达 $l=r$ 的时候,只剩下常数项,可以直接求值

对于 $X(x)$,可以先通过分治预处理出来

\section{多项式多点插值}

\begin{quotation}
    给定点集 $P=\{(x_i,y_i) | 0 \le i \le n\}$,求多项式 $f(x)$,满足 $\forall 0 \le i \le n, f(x_i)=y_i$
\end{quotation}

考虑拉格朗日插值:

$$
f(x)=\sum_{i=0}^{n}\frac{\prod_{j \not= i}(x-x_j)}{\prod_{j \not= i}(x_i-x_j)}y_i=\sum_{i=0}^{n}\frac{p_i}{q_i}y_i
$$

那么从分子和分母两部分进行考虑

考虑分母部分,设:

$$
M(x)=\prod_{i=0}^{n}(x-x_i)
$$

那么有:

$$
q_i=\frac{M(x)}{x-x_i}\Big|_{x_i}=\lim_{x \to x_i} \frac{M(x_i)}{x-x_i}=M'(x_i)
$$

设 $v_i=\frac{y_i}{\prod_{j \not= i}(x_i-x_j)}$,进一步化简 $f(x)$ 得到:

$$
f(x)=\sum_{i=0}^{n}v_ip_i=\sum_{i=0}^{n}v_i\prod_{j \not =i}(x-x_j)
$$

构造:

$$
\begin{cases}
X_l(x)=\prod_{i=0}^{\lfloor \frac{n}{2} \rfloor} (x-x_i) \\
X_r(x)=\prod_{i=\lfloor \frac{n}{2} \rfloor+1}^{n} (x-x_i) \\
\end{cases}
$$

则有:

$$
\begin{aligned}
f(x)
=&X_r(x)\left(\sum_{i=0}^{\lfloor \frac{n}{2} \rfloor} v_i \sum_{j \not=i,0 \le j \le \lfloor \frac{n}{2} \rfloor}(x-x_j) \right)+X_l(x)\left( \sum_{i=\lfloor \frac{n}{2} \rfloor+1}^{n}v_i \sum_{j \not= i, \lfloor \frac{n}{2} \rfloor+1 \le j \le n} (x-x_j) \right) \\
=&X_r(x)f_l(x)+X_l(x)f_r(x)
\end{aligned}
$$

依然是可以预处理 $X(x)$,对于 $f_l(x),f_r(x)$ 是可以递归求解的,在 $l=r$ 的时候返回 $v_l$

\section{常系数齐次线性递推}

\begin{quotation}
    设数列 $\{a_n\}$ 满足递推关系:

    $$
    a_n=\sum_{i=1}^{k}b_ia_{n-i}
    $$
    
    给定 $n$,求 $a_n$
\end{quotation}

\subsection{矩阵乘法}

一个比较简单的想法就是,构造转移矩阵,然后通过矩阵乘法以及快速幂实现求解

理论基础是矩阵乘法满足结合律,即 $(A \times B) \times C=A \times (B \times C)$,因此可以通过快速幂进行分治的求解

设 $M$ 为转移矩阵,$B$ 为初始矩阵,有:

$$
B=
\begin{bmatrix} 
    f_{k-1} \\
    f_{k-2} \\
    \vdots \\
    f_1 \\
    f_0 \\
\end{bmatrix}
$$

且:

$$
M=
\begin{bmatrix} 
    a _ 1 & a _ 2 & a _ 3 & \cdots & a _ {k - 2} & a _ {k - 1} & a _ k \\
    1 & 0 & 0 & \cdots & 0 & 0 & 0 \\
    0 & 1 & 0 & \cdots & 0 & 0 & 0 \\
    0 & 0 & 1 & \cdots & 0 & 0 & 0 \\
    \vdots & \vdots & \vdots & \ddots & \vdots & \vdots & \vdots \\
    0 & 0 & 0 & \cdots & 1 & 0 & 0 \\
    0 & 0& 0 & \cdots & 0 & 1 & 0
\end{bmatrix}
$$

那么如果要求 $f_{n}$,且 $n \ge k$,那么就相当于求 $(M^{n-k+1}B)_{0,0}$

唯一的缺点就是时间复杂度过高,达到了 $O(k^3 \log n)$,可以通过把这一类的转移矩阵转化为多项式来加速运算,达到 $O(k^2 \log n)$ 乃至 $O(k \log k \log n)$

\subsection{特征值和特征向量}

若有常数 $\lambda$,向量 $\vec v$,满足 $\lambda \vec v=A \vec v$ ,则 $\vec v$​ 为矩阵 $A$ 的一组特征向量,$\lambda$ 为矩阵 $A$ 的一组特征值

秩为 $k$ 的矩阵有 $k$ 组线性不相关的特征向量

\subsection{特征多项式}

$\lambda \vec v=A \vec v \Rightarrow (\lambda I-A)\vec v=0$,有解当且仅当 $\det(\lambda I-A)=0$

设 $f(\lambda)=\det(\lambda I-A)$,则 $f$ 是一个 $k$ 次多项式,同时 $\{\lambda|f(\lambda)=0\}$ 构成 $k$ 个特征值

于是可以把 $f(x)$ 写成 $f(x)=\prod_{i=1}^{k}(x-\lambda_i)$ 的形式

根据 \textbf{Cayley-Hamilton 定理},有 $f(A)=O$,其中 $O$ 是零矩阵

设转移矩阵为 $M$,实际上只需要求出 $M^n$ 就行了,先考虑怎么求 $f(x)$,根据定义,有:

$$
f(x) = |x I - M| = 
\det\left(
\begin{bmatrix} 
    x- a _ 1 & -a _ 2 & -a _ 3 & \cdots & -a _ {k - 2} & -a _ {k - 1} & -a _ k \\
    -1 & x & 0 & \cdots & 0 & 0 & 0 \\
    0 & -1 & x & \cdots & 0 & 0 & 0 \\
    0 & 0 & -1 & \cdots & 0 & 0 & 0 \\
    \vdots & \vdots & \vdots & \ddots & \vdots & \vdots & \vdots \\
    0 & 0 & 0 & \cdots & -1 & x & 0 \\
    0 & 0& 0 & \cdots & 0 & -1 & x
\end{bmatrix}
\right)
$$

直接根据行列式的定义,可以得到:

$$
f(\lambda)=\lambda^{k}-\sum_{i=1}^{k}a_i\lambda^{k-i}
$$

因为 $f(M)=O$,于是可以得到 $M^k=\sum_{i=1}^{k}a_iM^{k-i}$

\subsection{倍增求解}

设 $M^x=\sum_{i=0}^{k-1}a_iM^{i},M^y=\sum_{i=0}^{k-1}b_iM^{i}$,那么有:

$$
\begin{aligned}
M^{x+y}
&= \sum_{i=0}^{k-1}\sum_{j=0}^{k-1}a_iM^{i}b_jM^{j} \\
&= \sum_{c=0}^{2k-2}M^{c} \left(\sum_{i+j=c}a_ib_j\right)
\end{aligned}
$$

设 $M^{x+y}=f(M)g(M)+r(M)$,由于 $f(M)=0$,所以 $M^{x+y}=r(M)​$

\subsection{计算答案}

对于计算答案的时候,假设要计算 $M^nB$,展开后可以得到:

$$
M^nB=\sum_{i=0}^{k-1}a_i\left(M^{i}B\right)
$$

其中 $M^xB$ 对应着原序列的第 $x,x+1, \cdots, x+k-1$ 项,预处理 $f_0 \sim f_{2k}$ 即可计算

由于代码实现的原因,实际上在 $k=1$ 的时候要特殊处理,此时就是一个等比数列

\section{一阶线性微分方程}

\subsection{齐次线性方程}

\subsubsection{形式}

$$
\frac{dy}{dx}+P(x)y=0
$$

\subsubsection{求解}

$$
\begin{aligned}
& \frac{dy}{dx}+P(x)y=0 \\
\Rightarrow
& \frac{dy}{y}=-P(x)dx \\
\Rightarrow
& \int \frac{dy}{y}=c_1 -\int P(x)dx \\
\Rightarrow
& \ln |y|=c_1- \int P(x)dx \\
\Rightarrow
& y=C e^{-\int P(x)dx} \quad (C=\pm e^{c_1})
\end{aligned}
$$

\subsection{非齐次线性方程}

\subsubsection{形式}

$$
\frac{dy}{dx}+P(x)y=Q(x)
$$

\subsubsection{求解}

可以通过对应的齐次线性方程,使用 \textbf{常数变易法} 求解

设 $y=u(x) e^{-\int P(x)dx}$,则:

$$
\begin{aligned}
& \frac{dy}{dx}=\frac{du}{dx} e^{-\int P(x)dx} - u(x)P(x) e^{-\int P(x)dx} \\
\Rightarrow
& \frac{du}{dx} e^{-\int P(x)dx} - u(x)P(x) e^{-\int P(x)dx} + P(x)u(x) e^{-\int P(x)dx} =Q(x) \\
\Rightarrow
& \frac{du}{dx} = Q(x) e^{\int P(x)dx} \\
\Rightarrow
& \int du=C+\int Q(x) e^{\int P(x)dx} dx \\
\Rightarrow
& u(x)=C+\int Q(x) e^{\int P(x)dx} dx \\
\Rightarrow
& y= e^{-\int P(x)dx} \left( C + \int Q(x) e^{\int P(x)dx} dx \right) \\
\Rightarrow
& y= Ce^{-\int P(x)dx} + e^{-\int P(x)dx} \left( \int Q(x) e^{\int P(x)dx} dx \right)
\end{aligned}
$$

可以发现,等式右端第一项是对应的齐次线性方程的通解,第二项是非齐次线性方程的一个特解(在 $C=0$ 时取到)

即 \textbf{一阶非齐次线性方程的通解等于对应的齐次方程的通解与非齐次方程的一个特解之和}

\section{伯努利方程}

\subsection{形式}

$$
\frac{dy}{dx}+P(x)y=Q(x)y^{n} \quad (n \not\in \{0, 1\})
$$

\subsection{求解}

在 $n=0$ 的时候,退化成为了 \textbf{一阶齐次线性方程}

在 $n=1$ 的时候,退化成为了 \textbf{一阶非齐次线性方程}

首先有 $\frac{dy}{dx}+P(x)y=Q(x)y^{n} \Rightarrow y^{-n}\frac{dy}{dx}+P(x)y^{1-n}=Q(x)$

尝试用 $\frac{d}{dx}y^{1-n}$ 替换 $\frac{dy}{dx}$,得到 $(1-n)y^{-n}\frac{dy}{dx}$

为了消除 $1-n$,引入 $z=y^{1-n}$,则有 $\frac{dz}{dx}=(1-n)y^{-n}\frac{dy}{dx}$,代入原式可得:

$$
\begin{aligned}
& y^{-n}\frac{dy}{dx}+P(x)y^{1-n}=Q(x) \\
\Rightarrow
& \frac{dz}{dx}+(1-n)P(x)z=(1-n)Q(x) \\
\Rightarrow
& \frac{dz}{dx}+P_n(x)z=Q_n(x) \\
\end{aligned}
$$

那么就又转化为了 \textbf{一阶线性方程},代入公式可以得知:

$$
\begin{aligned}
z=e^{-\int (1-n)P(x)dx} \left( C + \int (1-n)Q(x) e^{\int (1-n)P(x)dx} dx \right) \\
\end{aligned}
$$

由于 $z=y^{1-n}$,所以 $y=z^{\frac{1}{1-n}}$,于是可以得出:

$$
y=\left(e^{-\int (1-n)P(x)dx} \left( C + \int (1-n)Q(x) e^{\int (1-n)P(x)dx} dx \right)\right)^{\frac{1}{1-n}}
$$

\section{黎卡提方程}

\subsection{形式}

$$
\frac{dy}{dx} = P(x)y^2+Q(x)y+R(x)
$$

一般形式暂时没有通用解法,但对于一些特殊形式还是可以解出来的

\subsection{$P(x),Q(x),R(x)$ 中至少两个同时为 $0$}

无非以下三种,均为 \textbf{变量分离方程}:

$$
\begin{aligned}
& \frac{dy}{dx} = P(x)y^2 \Rightarrow y^2dy=P(x)x \\
& \frac{dy}{dx} = Q(x)y \Rightarrow \frac{dy}{y}=Q(x)dx \\
& \frac{dy}{dx} = R(x) dy=R(x)dx \\
\end{aligned}
$$

\subsection{$\frac{dy}{dx} = Q(x)y+R(x)$}

退化为 \textbf{一阶线性方程}:

$$
\frac{dy}{dx} = Q(x)y+R(x)
$$

\subsection{$\frac{dy}{dx} = P(x)y^2+Q(x)y$}

退化为 \textbf{伯努利方程}:

$$
\frac{dy}{dx} = P(x)y^2+Q(x)y
$$

\subsection{$\frac{dy}{dx} + ay^2 = bx^m$}

需要保证:

$$
\begin{cases}
& a \not= 0 \\
& x \not= 0 \\
& y \not= 0 \\
& m = 0, -2, \frac{ -4k }{ 2k+1 }, \frac{ -4k }{ 2k-1 } ( k = 1, 2, \cdots ) \\
\end{cases}
$$

令 $x_1=ax$,则原式可以规约为 $\frac{dy}{dx} + y^2=bx^m$

当 $m=0$ 时,有:

$$
\begin{aligned}
& \frac{dy}{dx} + y^2 = b \\
\Rightarrow & \frac{dy}{dx} = b - y^2 \\
\Rightarrow & \frac{dy}{b-y^2} = dx \\
\Rightarrow & \int \frac{dy}{b-y^2} = C + x \\
\end{aligned}
$$

当 $m = -2$ 时,设 $z = xy$,有:

$$
\begin{cases}
& y=\frac{z}{x} \\
& \frac{dz}{dx} = \frac{ydx+xdy}{dx} = y + x\frac{dy}{dx} \\
\end{cases}
$$

于是有:

$$
\begin{aligned}
\frac{dz}{dx}
= & y + x\left(-y^2 + \frac{b}{x^2}\right) \\
= & \frac{z}{x} - \frac{z^2}{x} + \frac{b}{x} \\
= & \frac{b + z - z^2}{x} \\
\end{aligned}
$$

可以得到:

$$
\begin{aligned}
& \frac{dz}{b + z - z^2} = x dx \\
\Rightarrow & \int \frac{dz}{b + z - z^2} = C + \frac{x^2}{2} \\
\end{aligned}
$$

\subsection{$\frac{dy}{dx} + ay^2 = \frac{1}{x}y + \frac{b}{x^2}$}

设 $z=xy$,则 $y=\frac{z}{x}$,于是有:

$$
\begin{aligned}
& \frac{dy}{dx} + ay^2 = \frac{1}{x}y + \frac{b}{x^2} \\
\Rightarrow
& \frac{1}{x} \left(\frac{dz}{dx} - \frac{z}{x}\right) + a\frac{z^2}{x^2} = \frac{z}{x^2} + \frac{b}{x^2} \\
\Rightarrow
& \frac{1}{x} \frac{dz}{dx} = \frac{z}{x^2} - a\frac{z^2}{x^2} + \frac{z}{x^2} + \frac{b}{x^2} \\
\Rightarrow
& \frac{1}{x} \frac{dz}{dx} = \frac{2z - z^2 + b}{x^2} \\
\Rightarrow
& \int \frac{dz}{2z - x^2 + b} = C + \int \frac{xdx}{x^2} \\
\end{aligned}
$$

\section{参考文献}

\href{https://wenku.baidu.com/view/4b24054a16fc700abb68fc62.html}{黎卡提方程的初等解法}

\end{document}
