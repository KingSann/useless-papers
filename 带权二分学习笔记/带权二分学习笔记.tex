%!TEX program = xelatex

\documentclass[UTF8]{article}
\author {sjzez czy}
\title {带权二分学习笔记}
\date{2019.2.14}
\usepackage[UTF8]{ctex}

\usepackage{pgf,tikz,pgfplots}
\pgfplotsset{compat=1.15}
\usepackage{mathrsfs}
\usetikzlibrary{arrows}

\usepackage{listings}
\usepackage{fontspec}
\usepackage{amsmath}
\usepackage{amssymb}
\usepackage{geometry}
\usepackage{setspace}
\usepackage{abstract}
\usepackage{graphicx}
\usepackage{verbatim}
\usepackage[colorlinks,linkcolor=black,citecolor=black]{hyperref}
\renewcommand{\baselinestretch}{1.5}
\geometry{a4paper,left=2.7cm,right=2.7cm,top=2.7cm,bottom=2.7cm}
\setmonofont{Consolas}

\begin{document}

\maketitle

\begin{abstract}
    在一类最优化问题中,常有“恰好为 $k$ 个”之类的个数限制,如果直接通过动态规划解决的话,时间复杂度为 $O(nk)$ 左右,但在某些时候是可以根据函数的特殊性质降至 $O(n \log k)$
    \newline
    \centering
    \textbf{关键字:} 带权二分、凸函数、斜率、动态规划
\end{abstract}

\tableofcontents

\newpage

\section{框架}

设某个最优化模型如下:

\begin{quotation}

给定 $n$ 个物品,从中选择 $m$ 个物品使得收益最大

\end{quotation}

设 $g(x)$ 表示恰好选择 $x$ 个物品所得到的最大收益,目标是求出 $g(m)$,如果它是一个凸函数的话,则可以进行凸优化,为了简便起见,假设最优化的是最大收益,且 $g(x)$ 上凸

那么可以得到 $g'(x)$ 是一个单调递减的函数,且 $g(x)$ 的最大值在 $g'(x)=0$ 处取到,假设这个值为 $x_0$

那么可以二分一个权 $k$,表示每选择一个物品,那么会额外付出代价 $k$,构造函数 $f(x)=g(x)-kx$,由于 $g(x)$ 和 $-kx$ 都是凸的,因此 $f(x)$ 依旧是凸的,即 $f'(x)=0$ 的时候 $f(x)$ 取得最大值

假设 $f(x)$ 在 $f'(x)=g'(x)-k=0 \Leftrightarrow g'(x)=k$ 的时候取得最大值,此时 $x$ 为 $x_1$

由于 $g'(x)$ 是单调函数,所以随着 $k$ 的变化 $x_1$ 随着单调变化,即随着 $k$ 变小,$x_1$ 也会变大

换而言之,实际上是用一条斜率为 $k$ 的直线来切这个 $g(x)$ 构成的凸包

于是可以二分 $k$ 来得到 $x_1$,从而适当调整 $k$ 的取值

\definecolor{zzttqq}{rgb}{0.6,0.2,0}
\definecolor{xdxdff}{rgb}{0.49019607843137253,0.49019607843137253,1}
\definecolor{ududff}{rgb}{0.30196078431372547,0.30196078431372547,1}
\definecolor{uuuuuu}{rgb}{0.26666666666666666,0.26666666666666666,0.26666666666666666}
\begin{tikzpicture}[line cap=round,line join=round,>=triangle 45,x=1cm,y=1cm]
\begin{axis}[
x=1cm,y=1cm,
axis lines=middle,
ymajorgrids=true,
xmajorgrids=true,
xmin=-3.1000000000000005,
xmax=13.340000000000005,
ymin=-4.710000000000002,
ymax=9.150000000000004,
xtick={-3,-2,...,13},
ytick={-4,-3,...,9},]
\clip(-3.1,-4.71) rectangle (13.34,9.15);
\fill[line width=2pt,color=zzttqq,fill=zzttqq,fill opacity=0.10000000149011612] (0,0) -- (1,2) -- (2,3) -- (4,4) -- (7,3) -- (8,2) -- (9,0) -- cycle;
\draw [line width=2pt,color=zzttqq] (0,0)-- (1,2);
\draw [line width=2pt,color=zzttqq] (1,2)-- (2,3);
\draw [line width=2pt,color=zzttqq] (2,3)-- (4,4);
\draw [line width=2pt,color=zzttqq] (4,4)-- (7,3);
\draw [line width=2pt,color=zzttqq] (7,3)-- (8,2);
\draw [line width=2pt,color=zzttqq] (8,2)-- (9,0);
\draw [line width=2pt,color=zzttqq] (9,0)-- (0,0);
\begin{scriptsize}
\draw [fill=uuuuuu] (0,0) circle (2pt);
\draw[color=uuuuuu] (0.16,0.38) node {$A$};
\draw [fill=ududff] (1,2) circle (2.5pt);
\draw[color=ududff] (1.16,2.42) node {$B$};
\draw [fill=ududff] (2,3) circle (2.5pt);
\draw[color=ududff] (2.16,3.42) node {$C$};
\draw [fill=ududff] (4,4) circle (2.5pt);
\draw[color=ududff] (4.16,4.42) node {$E$};
\draw [fill=ududff] (7,3) circle (2.5pt);
\draw[color=ududff] (7.22,3.48) node {$H$};
\draw [fill=ududff] (8,2) circle (2.5pt);
\draw[color=ududff] (8.16,2.42) node {$I$};
\draw [fill=xdxdff] (9,0) circle (2.5pt);
\draw[color=xdxdff] (9.16,0.42) node {$J$};
\draw[color=zzttqq] (0.86,1.1) node {$a$};
\draw[color=zzttqq] (1.8,2.52) node {$b$};
\draw[color=zzttqq] (3.22,3.46) node {$c$};
\draw[color=zzttqq] (5.46,3.44) node {$d$};
\draw[color=zzttqq] (7.34,2.52) node {$e$};
\draw[color=zzttqq] (8.28,1.1) node {$f$};
\draw[color=zzttqq] (4.56,0.56) node {$g$};
\draw [fill=xdxdff] (2.996,3.498) circle (2.5pt);
\draw[color=xdxdff] (3.16,3.92) node {$D$};
\draw [fill=xdxdff] (4.993,3.669) circle (2.5pt);
\draw[color=xdxdff] (5.16,4.1) node {$F$};
\draw [fill=xdxdff] (6.007,3.331) circle (2.5pt);
\draw[color=xdxdff] (6.22,3.82) node {$G$};
\end{scriptsize}
\end{axis}
\end{tikzpicture}

\section{模型 A}

设某个最优化模型如下:

\begin{quotation}

给定 $n$ 个物品,从中选择 $m$ 个物品使得收益最大,其中 $g(x)$ 上凸

\end{quotation}

那么二分的 $k$ 的取值范围是 $(-\infty, \infty)$,在 $x_1 \le m$ 的时候记录 $k$,同时将 $k$ 往小调整,否则将 $k$ 往大调整

输出答案的时候用最后一次记录的 $k$ 进行计算

\section{模型 B}

设某个最优化模型如下:

\begin{quotation}

给定 $n$ 个物品,从中选择 \textbf{最多} $m$ 个物品使得收益最大,其中 $g(x)$ 上凸

\end{quotation}

由于是“最多”的限制,因此不能再像之前一样设置二分边界了,此时应该设为 $[0, \infty)$

此时当 $m \le x_0$ 的时候,最优解斜率在二分边界之内,因此可行,当 $m > x_0$ 的时候,最优解应在 $x=x_0$ 处取到,此时最优解的斜率为 $0$,依然在二分边界之内

\section{例题}

\href{https://www.luogu.org/problemnew/show/P1792}{种树 A}

\href{https://www.luogu.org/problemnew/show/P1484}{种树 B}

\end{document}